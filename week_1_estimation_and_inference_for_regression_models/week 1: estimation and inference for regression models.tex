%! Author = noone
%! Date = 5/3/23

% Preamble
\documentclass[11pt]{article}

% Packages
\usepackage{amsmath}
\usepackage{blindtext}
\usepackage{amsfonts}

% Document
\title{Sections and Chapters}
\author{Overleaf}
\date{\today}

\begin{document}
\maketitle
\section{What is Regression analysis?}\label{sec:what-is-regression-analysis?}

Regression Analysis investigates the functional relationship between statistical variables.
Data are usually multiple observations of a random vector $(Y,x)$.
\begin{itemize}
\item $x=(X1,\dots,Xp)^T$ is a $p$-vector of variables termed: \underline{explanatory variables}, regressors, predictors, input variables or \underline{independent variables}.
\item $Y$ is called: response variable, target variable, output variable, outcome variable or dependent variable.
It may be continuous $(\in \mathbb{R})$, discrete $(\in \{1,\dots,K\} \in \{1,\dots,K\})$ or ordinal (ordered discrete).
\end{itemize}
Response variables are usually treated as \textbf{random variables}, while predictors are treated as \textbf{fixed observations}.
\blindtext

\section{Response and explanatory variables}\label{sec:response-and-explanatory-variables}
Response and explanatory variables

Response and explanatory variables are measures on one of the following scales:
\begin{itemize}
\item \textbf{nominal}: when $Y$ is classified into categories, which can be only two (binary outcome) or several (multinomial outcome)
\item \textbf{ordinal}: when $Y$ is recorded in classes
\item \textbf{continuous}: when $Y$ is measured on a continuous scale, at least in theory.
\end{itemize}
Nominal and ordinal data are discrete variables and can be qualitative or quantitative (eg \text{counts}).
Continuous data are quantitative.
$x$ can also be quantitative or qualitative.
In particular, when the explanatory variable is qualitative, it is often called factor.
A quantitative explanatory variables is called covariate.

\blindtext

\section{Regression}\label{sec:Regression}


\blindtext

\section{applied regression analysis}\label{sec:applied regression analysis}

\blindtext

\section{General Framework of statistical learning}\label{sec:General-framework-of-statistical-learning}

\blindtext

\section{Knowledge Assumed}\label{sec:Knowledge-Assumed}

\blindtext

\end{document}
